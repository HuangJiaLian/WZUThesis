
\chapter[致谢]{致\quad 谢}
\vbox{}

三年光阴,稍瞬即逝,硕士生涯的结束,伴随着新的人生旅程的开始,不断的感叹时间走的匆忙,却又不断地继续砥砺前行。第一次来校的画面依旧清晰,却转眼即将又要离去,人生有太多结束,也有太多的开始。开始的终究要结束,抱着一颗平常心、感恩的心对那些曾经给予过帮助的人和单位致以衷心的感谢!

首先,我要感谢我有一个强大的、安定的祖国,是祖国的昌盛给予了我们一切,是祖国的安定给予了我们安定的生活,是祖国的繁荣给予了我们想要的一切,真心的祝福我的祖国未来越来越好!也祝愿人们的生活质量越来越高,不要再有穷人,不要再有坏人,不要再有流浪汉!

其次,我要感谢我的导师王明锋老师,感谢王老师这三年来对我悉心教导,王老师不仅是我学术生涯的领路人,同样也是我的人生导师更是生活中的朋友,他不仅教会了我如何做研究,也教会了我很多做人的道理。在这里我祝王老师家庭和睦,身体健康,工作顺利!

再次,我要感谢温州大学那些给我授过课的老师,感谢你们的辛勤付出!也感谢物电学院所有的在我学习和生活上给予过我帮助和关怀的老师,特别感谢何林李老师、姜年权老师、朱海永老师、林莘莘老师、王艳伟老师!也感谢16物研班的所有同学,相遇便是缘,谢谢你们的陪伴!

从次,我要感谢在硕士期间陪伴过我度过三年时光的研究生同学,是你们让我的生活不再孤单,特别感谢刘姗姗同学!也感谢我认识的部分本科生,你们真的比研究生有趣太多太多,谢谢你们给我的生活增添了更多的色彩!也感谢体育学院的蔡老师,做朋友没的说!

最后,我要感谢我的父母,感谢你们养育了我,感谢你们为我付出这么多!从小到大,你们为我付出了太多太多,而我现在却什么都做不了,只希望自己可以早点工作赚钱,让你们早日过上好日子!

除此之外,我还要感谢以下的人:
\begin{itemize}[itemsep=0pt,topsep=0pt]
	\item 感谢LaTeX群群友在我论文排版过程中对我提供的无私的帮助!
	\item 感谢MMA群群友在我写论文过程中对我技术上提供的指导与帮助,特别感谢南京理工大学的王景弘同学!
%	\item 感谢我那一直从未出现的女朋友,谢谢你没有来打扰我写论文!
	\item 感谢那些在我生命中出现的帮助过我的所有人,谢谢你们!
\end{itemize}

其实,我最最需要感谢的那个人是我自己!

%调整 LaTeX 中的列表环境时,使用 enumitem 宏包可以方便的调整间距。
%调整间距的参数命令包括两类。

%-垂直间距
%   topsep             列表环境与上文之间的距离
%   parsep             条目里面段落之间的距离
%   itemsep            条目之间的距离
%   partopsep          条目与下面段落的距离
%-水平间距
%   leftmargin         列表环境左边的空白长度
%   rightmargin        列表环境右边的空白长度
%   labelsep           标号与列表环境左侧的距离
%   itemindent         条目的缩进距离
%   labelwidth         标号的宽度
%   listparindent      条目下面段落的缩进距离

%   对给予各类资助、指导和协助完成研究工作以及提供各种对论文工作有利条件的单位及个人表示感谢。
%   致谢应实事求是,切忌浮夸与庸俗之词。限一页。

%   三年的硕士生活,不经意间已悄然结束,每次到了说再见的时候,都会象征性的总结一番。
%   人生百年岁月,已然过去四分之一,儿时伙伴已为人父母,而我依旧在校园里读着“圣贤书”。
%   到了我这个年纪,这个尴尬的年纪,本不应该再有迷茫,然而,迷茫从未停止过。
%   回想已逝时光,竟难让人有些慌神。
%   感叹人生百年岁月,过的如此之快,时间又是如此的短暂。
%   硕士三年生活,即将告一段落,也许,我的学生生涯也即将终结,但终结并不意味着结束,它只是一个新的起点!

%   其实,我是应该好好写一篇总结的,把它当做是对我三年研究生生涯的一个交代!

\chapter{攻读硕士期间发表的论文}
\vbox{}

\hspace{-0.85cm}[1] 朱晴羽, 刘刚, 吴妙鑫, 郑亦庄, 利用光子回声技术对光的相干态进行存储的研究, 量子光学学报, 2017(05).

\hspace{-0.85cm}[2] Gang Liu, YaNi Wang, LiFen Yan, NianQuan Jiang, Wei Xiong and MingFeng Wang$^\dagger$, Spin squeezing via one- and two-axis twisting induced by a single off-resonance stimulated Raman scattering in a cavity, Phys. Rev. A 99. 043840(2019).

\hspace{-0.85cm}[3] 刘刚, 王娅妮, 王明锋, 利用相干光场制备原子系综纠缠态的研究, 量子光学学报. (审稿中)

