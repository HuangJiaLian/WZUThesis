
\chapter{绪论}\label{chapter1}
\vbox{}\vbox{}
\section{引言}\label{section11}
\vbox{}
近些年来,量子信息\cite{bennett1993teleporting,bennett1992communication,bennett1984proceedings,beige2001secure}和量子计算\cite{shor1994proc,nielsen2002quantum,ekert1996quantum}研究领域的兴起正引起人们的浓厚兴趣,因为其在人们生活中有着很重要的应用。量子信息和量子计算之所以备受人们的青睐,主要在于它们拥有一些经典通信和计算所不具备的特性,比如量子算法和量子隐形传态等\cite{bennett1992communication},而这些功能的实现,与量子纠缠有着密切的联系。能够在理论上研究清楚量子纠缠到底是什么,这已经成为这个领域的一个核心问题。从1935年EPR佯谬\cite{einstein1935can}和薛定谔的猫以来\cite{article1935},无论是实验物理学家还是理论物理学家他们对量子纠缠的研究都非常的感兴趣。量子纠缠之所以这么“受欢迎”,主要是因为它有一些经典关联所没有的特性,而正因为这些特性存在所以使得量子纠缠在量子隐形传态和量子计算等方面都占有核心地位。但是截至目前为止如何有效的描述量子纠缠一直都显得格外的困难,这与量子纠缠的特殊性有着很重要的关系。目前已有的比较成功的描述两qubit纠缠的量是所谓的并发度(concurrence)\cite{article1997,PhysRevA.68.012101}描述,但是对于多体纠缠的描述一直以来都没有找到一个好的解决方案。

除了其在量子通信和计算方面的用途以外,科学家们还发现利用纠缠可以提高量子精密测量的测量精度,他们在研究干涉仪和原子钟等问题的时候,发现了一种可以能够使实验上相位和频率的测量相对于无关联态时提高的态,通过研究表明其正是我们接下来要讲的自旋压缩态,我们用自旋压缩参数来描述它。

由于当时对纠缠的认识上的欠缺,因而并没能特别指出自旋压缩态和纠缠之间的关系。后来随着研究的进一步的深入,人们发现自旋压缩和纠缠之间有着一种天然的联系,而自旋压缩参数除了用来描述自旋压缩之外,还可以用来判据纠缠。自旋压缩参数较其它的纠缠判据(如并发度)相比有如下的优点:首先,自旋压缩参数的定义与角动量算符的平均值和涨落性质有关,因而有比较清晰的物理图像;其次,自旋压缩态与量子精密测量和原子钟等之间有着十分紧密的关系,因而有更加实际的应用。

\vbox{}
\section{自旋压缩理论的研究背景}\label{section12}
\vbox{}
压缩态概念的最早出现是Earle Kennard在1927年发表的论文\cite{Kennard1927}(尽管当时没有创造出压缩态),他在海森堡不确定关系的约束下处理了谐波振子的一般高斯波包的演化,从而产生了压缩态。后来Stoler在前人的基础上于1970年提出了光场压缩态的概念\cite{stoler1970equivalence}。1976年,Yuen在研究了压缩态光场的量子特性\cite{yuen1976two}以后提出了双光子相干态的概念,实质是压缩态。
人们在研究光场压缩性质的时候,自然而然的就想到在光与原子系综相互作用的系统中,原子系综是不是也会出现压缩性质?于是人们开始了对原子压缩的研究。

1981年,Walls和Zoller第一次将光场压缩的概念推广到原子系统中\cite{PhysRevLett.47.709},引入了原子偶极压缩的概念。后来人们对原子系综的压缩进行了不断深入的研究,结果发现在自旋系统或角动量系统中也有着类似的压缩性质存在,于是自旋压缩态的研究正式拉开序幕。

1992年,Wineland等人\cite{PRA1992Wineland}发现,当原子系综处于压缩态的时候,其系统的量子噪声会减少,从而可以提高光谱的灵敏度,由于这个现象和光场压缩态的很相似,所以相应的量子态就叫做原子的自旋压缩态。后来研究表明,要想在一个集体自旋系统中产生自旋压缩态,则一定要有非线性耦合才可以,因为线性耦合只是对系统进行一个旋转操作。

1993年,Kitagawa和Ueda\cite{PRA1993Kitagawa}在理论上利用两个典型的非线性耦合(单轴扭曲和双轴扭曲)产生了自旋压缩态。自此以后,一些其他的产生自旋压缩态方法也相继被提出。系统能够实现自旋压缩也就意味着系统中存在着纠缠,同时也就意味着可以实现更高精度的测量,因而如何产生自旋压缩态在量子信息和量子计算中也有着非常重要的意义。

2002年,S\o{}rensen和M\o{}lmer等人\cite{PRA2002SS}通过将大量铷原子放在坏腔中利用光与原子相互作用产生自旋压缩态。研究表明,由于原子衰变和腔损失引起的粒子之间的量子相关性的损失可以通过在大量原子系统中的强非线性耦合来平衡,所以在坏腔与场的耦合中有可能观察到显著的原子自旋压缩。2010年,Anne Louchet-Chauvet等人\cite{LCA2010}使用量子非破坏性测量的方法来产生自旋压缩态,并在$10^5$个冷铯原子云中形成纠缠。他们的实验方案较之前的原子投影噪声极限相比,其信噪比提高了1.1dB。2017年,J. Borregaard等人\cite{NJP2017J-Borregaard}在\cite{PRA2002SS}的基础上,另外加了两束激光,利用强光控制弱光的方法将系统的哈密顿量就变成了一个双轴扭曲的哈密顿量,此时产生了较前面方案更强的自旋压缩态。

以上各方法作为自旋压缩态的产生研究对量子信息的发展都起到了很大的推动作用,但作为本论文的选题,需要对其的优缺点进行分析和评价,然后选择一个合适的突破点作为本论文的切入点。在光学腔中利用光与原子相互作用来产生自旋压缩的方案最为成熟,但是已有的理论方案过于复杂,由于量子系统的特殊性,方案过于复杂不利于在实验中产生较纯压缩度较高的自旋压缩态。于是进一步的改进并简化理论方案,优化处理物理过程的数学手段,成了其发展过程至关重要的一个环节。

\vbox{}
\section{自旋压缩态的主要应用}\label{section13}
\vbox{}
自旋压缩态(Squeezed spin states)是集体自旋的纠缠量子态,由于海森堡不确定关系给测量仪器设置了标准量子极限,无论测量仪器多么精密,误差始终存在,但是处于自旋压缩态的原子系综的两个不对易的相互正交的自旋分量可以在不违反海森堡不确定度关系的前提下,通过增大一个自旋分量的涨落来降低另外一个与之不对易的自旋分量的量子涨落从而使其低于真空涨落,这就使得自旋压缩态研究变得至关重要。由于产生自旋压缩的具体系统的不同,所以自旋压缩的定义并不唯一,相应也有许多不同的应用,主要包括两个方面:

自旋压缩的一个主要应用是用来检测量子纠缠,量子纠缠在量子物理学的基础和量子信息处理中都起着关键作用。 参数$\xi_s^2$与负成对关联和并发度相关,可以表征一类多体自旋1/2态的成对纠缠。自旋压缩参数是多体纠缠的证据判据。

自旋压缩的另一个应用,是在实验中提高测量的精度,例如在Ramsey光谱学中,以及制造更精确的原子钟和重力波干涉仪。 因此,许多努力都致力于在原子系统中实现压缩。 这些工作基本上可以分为两类:通过原子-光子相互作用在原子系综中产生自旋压缩和通过原子碰撞在BEC中产生自旋压缩。

\vbox{}
\section{本文研究的内容}\label{section14}
\vbox{}
本文的研究主要围绕在光学腔中$\Lambda$型原子与量子光场相互作用的问题展开,研究了原子系综在经典光场的驱动下,其压缩性质的变化以及不同原子系综之间纠缠的的产生问题。

第一章,我们先介绍了自旋压缩态的研究背景、研究意义、研究现状及其它的主要应用,然后介绍了本文的安排。

第二章,我们介绍了与本文有关的一些基本理论知识。首先介绍了光场的量子化,光与原子的相互作用等,然后我们介绍了自旋压缩态的概念、判据以及产生等知识,再后我们也介绍了关于纠缠的一些理论,包括纠缠的概念、纠缠的分类以及应用,最后我们也介绍了海森堡朗之万理论的知识。

第三章,我们提出了在腔QED系统中通过一个非共振受激拉曼过程来制备原子系综自旋压缩态的理论方案。原子系综在光场的驱动下,通过选取合适的参数得到了一个OAT型的有效哈密顿量,随后我们研究了在噪声的影响下这个哈密顿量是如何支配系统的演化的,在得出演化方程以后我们分析了系统的压缩特性,通过分析表明我们的方案可以获得不菲的压缩度。我们通过添加一个沿极化方向的磁场发现可以将OAT压缩转变为TAT压缩,虽然压缩度有所提高,但其对噪声的影响也更加敏感。

第四章,我们提出了利用相干光场来制备原子系综纠缠态的方案。该方案是基于置于光学腔中的原子系综被相干光场驱动,利用腔模的“量子公共汽车”效应来建立起原子之间的非经典关联的研究。两个原子系综在相干光场的驱动下,通过将得到的系统哈密顿量消除激发态从而得到有效哈密顿量,然后在Holstein-Primakoff近似下,定义新的原子算符,并计算出了其在系统哈密顿量作用下随时间的演化方程。利用Simon的Peres-Horodecki判据我们证明了在这个哈密顿量支配下,两个原子系统之间的确有纠缠发生。我们也分析了存在噪声的情况下我们方案的性能。

第五章,我们对全文的一个总结与展望。





%具体以单模光场与多个原子相互作用的模型为例,我们首先研究了在光学腔中量子光场与原子之间通过非共振受激拉曼过程可以产生自旋压缩态,随后我们详细的分析了原子系综中自旋压缩的动力学特性,并分析了自旋压缩的一些特性。随后我们在同样的光学腔中通过光与原子相互作用建立了两个原子子系统之间的纠缠,并分析了纠缠结果。本文共分为六章,各章内容介绍如下:

% 在这一部分我们主要研究了同样在光学腔中如何在两个原子系综的子系统之间建立纠缠的问题。我们研究发现,在光学腔内部放入两堆原子(系综),则这两个系综会在强的驱动场的作用下彼此之间会建立起纠缠。我们先写出了系统的哈密顿量,然后在大失谐的情况下绝热消除掉激发态,最后我们得到了在某个旋转框架下的哈密顿量,在Holstein-Primakoff近似下,我们定义了新的原子算符,并计算出了其在系统哈密顿量的作用下随时间的演化。我们利用Simon的Peres-Horodecki判据证明了在我们的这个哈密顿量下,两个子系统之间的确有纠缠发生。我们希望我们的方案能够在量子信息或量子计算的发展进程中起到一点积极的意义。


%主要研究了在光学腔中如何产生自旋压缩态。我们通过研究表明,如果将三能级原子系综置于光学腔中,然后向其内部注入一束强激光,则原子系综会在强光场和腔模共同作用下产生压缩特性。
%通过分析这些压缩特性我们发现其是单轴扭曲型的压缩,附加适当的磁场可以将单轴扭曲的哈密顿量转变成双轴扭曲的哈密顿量,从而可以提高系统的压缩度。我们还分析了噪声对系统的影响,发现即便在存在噪声的情况下,系统也可以产生很高的压缩度。

%研究了在光学腔中如何建立两个原子系综之间的纠缠。在同一光学腔内放入两个原子系综,通过研究表明,在光学腔内部放入两个原子系综,则这两个系综会在强驱动场的作用下彼此之间建立纠缠。这一纠缠根植于原子之间的非破坏性相互作用,利用Simon的Peres-Horodecki判据证明了在这个哈密顿量下,两个原子系综之间的确存在着非经典关联。而非破坏性相互作用是连续变量量子计算中一个很重要的量子门,故我们相信本方案对量子计算的发展也起到一定的作用。





%第一章,简要的介绍了一下自旋压缩理论的研究背景,研究进展以及它的主要应用。第二章,主要系统地介绍了一些与本文相关的理论知识。第三章,详细介绍了自旋压缩的定义,自旋压缩的判据以及纠缠的相关知识。第四章,我们研究了在光学腔中通过非共振的受激拉曼过程来实现OAT和TAT的自旋压缩。第五章,我们研究了如何在光学腔中通过光与原子相互作用来产生两个原子系综之间的纠缠。第六章,是对本文的总结与展望。

