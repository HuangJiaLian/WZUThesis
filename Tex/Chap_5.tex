%
\chapter{结论与展望}\label{chapter5}
\vbox{}\vbox{}
\section{结论}
  量子纠缠是一种很宝贵的资源,可以用来验证量子力学中的基本问题,在量子信息中也有很重要的应用。它和自旋压缩作为量子信息处理的基本资源,在量子光学领域占有很重要的地位。
  原子的自旋压缩的应用极为广泛,不光可以用来产生多体纠缠,在精密测量和量子信息的存储等方面也有着很重要的应用。  
  本文主要研究了$\Lambda$型三能级原子在单模光场的驱动下原子系综的自旋压缩特性和两个原子系综之间的纠缠特性。
  
  首先我们研究了光学腔中原子系综的自旋压缩特性,通过分析这些压缩特性我们发现其是单轴扭曲型的压缩,附加适当的磁场可以将单轴扭曲的哈密顿量转变成双轴扭曲的哈密顿量,从而可以提高系统的压缩度。我们还分析了噪声对系统的影响,发现即便在存在噪声的情况下,系统也可以产生很高的压缩度。
  
  然后我们研究了在光学腔中如何建立两个原子系综之间的纠缠的问题,在同一光学腔内放入两个原子系综,通过研究表明这两个系综会在强驱动场的作用下彼此之间建立纠缠。这一纠缠根植于原子之间的非破坏性相互作用,利用Simon的Peres-Horodecki判据证明了在这个哈密顿量下,两个原子系综之间的确存在着非经典关联。我们也研究了噪声对纠缠的影响,研究表明在不存在噪声的影响下,系统的纠缠随着相互作用的强度的增大而逐渐增强。当噪声大于1/2以后,原子之间的纠缠完全被系统的噪声所淹没。而当噪声在0和0.5之间的时候,纠缠由于噪声的影响会先变大后减小。
  
\vspace{0.5cm}
  \section{展望}
  量子信息科学目前正在进行着飞速的发展,关于纠缠和自旋压缩的研究目前也有很多,都取得很不错的成果,部分成果都已经被用到了实际的生活中,但是由于退相干的影响,实际的效果虽然较经典理论的应用有了很大的进步,但是其发展空间依旧还很大。如何实现多量子比特的纠缠和如何有效的克服系统的噪声来实现更高效的压缩这是人们目前致力于攻克的两个难题。

  在原子系综的压缩方面,迄今为止研究的焦点都主要是集中在如何增强原子间的纠缠上,而忽视了原子本身实际上也是一个小系综(例如铯原子的F=4态可等效8个1/2自旋原子),显然,构建原子间纠缠的同时压缩原子内态(即单个原子形成的小系综),必然会增大体系整体的压缩度。如果我们研究对内外态同时作用不同的哈密顿量,观测体系压缩度增加的情况,预计在内外态同时进行压缩的情况下,体系的压缩度将得到大大的提高。与此同时如果我们可以建立一个超真空的环境,在这个环境下对体系进行相应的操作,也许体系受到的环境等其他因素带来的噪声可以大大的降低。
  
    